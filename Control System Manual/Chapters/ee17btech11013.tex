\begin{enumerate}[label=\thesection.\arabic*.,ref=\thesection.\theenumi]
\numberwithin{equation}{enumi}

\begin{frame}
   A lead Compensator network includes a parallel combination of R and C in feed-forward path. If the transfer function of compensator  is \begin{align}G_c(s) = \frac{s+2}{s+4}\end{align}, the value of RC is ?\\
   And also find the value of RC for a lead compensator used in previuos example. \begin{align}G_c(s) = \frac{3(s+\frac{1}{3})}{s+1}\end{align}
\begin{figure}[!ht]
    \begin{center}
    {% \begin{figure}
\begin{circuitikz}[american voltages]
\draw
  (0,0) to [short, *-](1,0) to [resistor, l=$R_1$] (5.5,0)
  (1, 0) -- (1, 1.5) to [capacitor, *-*, l=$C$] (5.5, 1.5) -- (5.5, 0) -- (6,0) to [resistor,, l=$R_2$] (6,-3) to [short, -*] (0, -3) 
  (6, 0) to [short, -*] (7, 0) 
  (6, -3) to [short, -*] (7, -3);
  \end{circuitikz}
% \end{figure}
}
    \end{center}
\caption{}
\label{fig:ee18btech11005}
\end{figure}

Solution:\\
The transfer function for the following circuit is
    \begin{align}
    T(s) = \frac{V_o}{V_i}
    \end{align}
    Let
    \begin{align}
    \alpha = \frac{R_2}{R_1 + R_2}
    \end{align}
    and 
    \begin{align}
    \tau = R_1C
    \end{align}

Now our T(s) is 
    \begin{align}
    T(s) = \frac{R_2}{\frac{\frac{1}{sC}R1}{\frac{1}{sC}+R1} + R2}
    \end{align}
    Simplifying T(s)
    \begin{align}
    T(s) = \frac{s+\frac{1}{\tau}}{s+\frac{1}{\tau\alpha}}
    \end{align}
    Comparing with the given $$G_c(s) = \frac{s+2}{s+4}$$
    \begin{align}
    \tau = R_1C = 0.5
    \end{align}
    for
    \begin{align}
    T(s) = \frac{3(s+\frac{1}{3})}{s+1}
    \end{align}
    here this is a lead compensator with a gain of 3.
    so we can simply write passive circuit part as.
     \begin{align}
    T(s) = \frac{(s+\frac{1}{3})}{s+1}
    \end{align}
    again by comparing with 
    \begin{align}
    T(s) = \frac{s+\frac{1}{\tau}}{s+\frac{1}{\tau\alpha}}
    \end{align}
    
    \begin{align}
    \tau = 3
    \end{align}
    \begin{align}
    RC = 3
    \end{align}
\end{frame}

\end{enumerate}
